从实验结果可以看到,编译器的报错信息和我们的预期不总是一致的。但是编译器的报错是根据语法分析的结果,也是合理的。
事实上,对于实现完整符号表的前端,第4个错误程序报错信息应该是类似于未定义标识符,因为该变量没有加入到符号表中,符号表无法查询到相应的标识符;而对于第6个报错信息也应该是而定义标识符,因为此时处于对执行语句表解析的阶段,应该分析到一个赋值语句,但是readm也是一个没有加入到符号表的标识符,因此无法识别;

对于我们这个没有实现完整符号表的编译器前端,我们的报错结果是符合没有符号表的语法分析的结果的,并不完全是错误的;

代码的展示如附录一所示;

