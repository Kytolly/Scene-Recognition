实验结果显示,第二种方法(BOW + SVM)的准确率
远高于第一种方法(Tiny Image + KNN)的准确率。
推测原因如下:

Tiny Image 特征是非常简单的全局特征,仅通过将图片缩放到非常小的尺寸来表示。
这种方法丢失了图像的绝大部分细节信息,对图像的尺度、平移、旋转以及视角变化非常敏感。
它只能捕获非常粗略的空间布局和颜色信息,难以区分细节丰富的不同场景。
Bags of Words 特征是基于局部特征描述符(如 HOG 或 SIFT)构建的。
这些局部描述符(如 SIFT)能够捕捉图像中具有区分性的局部模式
(如边缘、角点、纹理等),并且对局部的光照、尺度和旋转变化具有一定的鲁棒性。
将这些局部特征聚类形成"视觉词汇",并将图像表示为这些视觉词汇的直方图,
能够更有效地捕获场景的构成元素及其分布,
从而提供了比 Tiny Image 更丰富、更具判别力的图像表示。


KNN分类器是一种基于实例的学习方法,
它直接根据测试样本与训练样本在特征空间中的距离进行分类。
虽然简单直观,但 KNN 对噪声和不相关的特征比较敏感,
并且在处理高维稀疏数据(如 Bag of Words 直方图)时,
"距离"的定义和计算可能会受到"维度灾难"的影响,性能可能受限。
SVM是一种功能强大的分类器,
尤其擅长处理高维数据。
SVM 尝试在特征空间中找到一个最优的超平面来最大化不同类别之间的间隔。
线性 SVM(LinearSVC)虽然是线性的,
但它在高维空间中表现良好,并且通过合适的正则化可以有效地防止过拟合。
相比于 KNN 仅依赖局部少数样本,SVM 学习的是一个全局的决策边界,
这通常在高维复杂的特征空间中更具优势。

综上所述,Bags of Words 方法通过更鲁棒、
更具判别力的局部特征表示捕获了场景的关键信息,
而 SVM 分类器则能够有效地利用这种高维特征进行分类。
因此,BOW + SVM 的组合在场景分类任务上取得了显著优于 Tiny Image + KNN 方法的性能。
