尽管实验成功验证了核心原理并构建了基本框架,但也认识到当前实现作为生产级编译器前端存在诸多不足(如错误恢复的健壮性、符号表中信息的完整性、对复杂文法结构的处理能力、性能优化以及与现代编译器架构的差距等)。

总而言之,本次实验通过从零开始实现一个简单的编译器前端,不仅成功地将编译原理的理论知识转化为实际代码。 通过构建了一个可工作的简单demo,我加深了对词法分析、语法分析、作用域管理和符号表工作原理的理解。实验结果证明了所采用方法的有效性。