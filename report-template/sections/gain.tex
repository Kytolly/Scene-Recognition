我深入理解了基于传统特征和机器学习方法的图像场景分类流程,包括特征提取、特征表示(Bag of Words 模型)和分类器设计。
掌握了 Tiny Image 特征和局部描述符(如 HOG/SIFT)的基本原理及其在场景识别中的应用,认识到不同特征对最终性能的影响。

我实践了聚类算法在构建视觉词汇中的作用,理解了 Bag of Words 模型如何将变长的局部特征序列转化为固定长度的图像表示向量。
对比了基于实例的 KNN 分类器和基于决策边界的 SVM 分类器在处理图像特征时的特点和性能差异,认识到 SVM 在高维特征空间中的优势。

通过本次场景分类实验,我对经典计算机视觉方法有了更直观的认识,为其在现代深度学习方法中的演进奠定了基础。
