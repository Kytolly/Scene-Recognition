\section{K近邻分类器实验原理}

K近邻(K-Nearest Neighbors, KNN)是一种简单直观的监督学习算法,
主要用于分类和回归任务。
在本实验中,它被用作分类器,特别是与Tiny Images特征结合使用。

\subsection{基本原理}
KNN的核心思想是:一个样本的类别由其在特征空间中的K个最近邻居的类别决定。

\subsection{KNN特点}
\begin{itemize}
    \item \textbf{简单易懂}:原理直观,易于实现
    \item \textbf{无需训练}:仅存储训练数据,无复杂训练过程
    \item \textbf{计算开销}:预测阶段需计算与所有训练样本的距离,大数据集上耗时
    \item \textbf{对噪声敏感}:K值较小时对噪声和离群点敏感
    \item \textbf{维数灾难}:高维特征空间中距离计算意义减弱,数据稀疏
    \item \textbf{非参数模型}:不对数据分布做假设
\end{itemize}

\subsection{算法步骤}
算法流程如下伪代码所示
\begin{algorithm}[H]
  \caption{Pseudocode for K-Nearest Neighbors (KNN) Classification}
  \begin{algorithmic}[1]
    \Require Test data feature vector $x_{\text{test}}$,Training data feature vectors $\text{TrainingData} = \{x_1, x_2, ..., x_n\}$,Corresponding training labels $\text{TrainingLabels} = \{y_1, y_2, ..., y_n\}$,Number of nearest neighbors $K$
    \Ensure Predicted class label $y_{\text{pred}}$ for $x_{\text{test}}$
    \State Initialize an empty list: $\text{NeighborsList}$ (to store Distance, Label tuples)

    \For{each training sample $x_i$ in $\text{TrainingData}$}
      \State Calculate the distance between $x_{\text{test}}$ and $x_i$: $Distance_i \leftarrow \text{Distance}(x_{\text{test}}, x_i)$
      \State Get the corresponding label $y_i$ from $\text{TrainingLabels}$
      \State Add tuple $(Distance_i, y_i)$ to $\text{NeighborsList}$
    \EndFor

    \State Sort $\text{NeighborsList}$ based on distance in ascending order

    \State Select the first $K$ elements from sorted $\text{NeighborsList}$ as $\text{K\_Nearest\_Neighbors}$

    \State Initialize an empty map: $\text{LabelCounts}$ (to count frequency of labels)

    \For{each neighbor $(distance, label)$ in $\text{K\_Nearest\_Neighbors}$}
      \State Increment count for $\text{label}$ in $\text{LabelCounts}$
    \EndFor

    \State Find the label with the maximum count in $\text{LabelCounts}$
    \State $y_{\text{pred}} \leftarrow \text{Label with highest count in } \text{LabelCounts}$
    \Comment{Handle ties if necessary}

    \State \Return $y_{\text{pred}}$
  \end{algorithmic}
\end{algorithm}


\subsection{K值选择}
参数K是KNN算法最重要的决定因素:
\begin{itemize}
    \item K=1时,新样本类别由最近训练样本决定,对噪声敏感
    \item 增大K可减少噪声影响,使决策边界更平滑
    \item K过大可能包含不相关邻居,导致性能下降
    \item 通常通过交叉验证选择最优K值
\end{itemize}



