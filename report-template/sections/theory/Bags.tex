\subsection{Bag of Words (BoW) 实验原理}

Bag of Words (BoW) 模型最初用于文本分析,
将文档表示为其包含的单词的频率直方图,忽略词语的顺序。
将这个思想迁移到图像领域,就是将图像视为"视觉词汇"的集合。
BoW模型通过提取局部特征、构建视觉词汇表、生成特征直方图等步骤,
实现了对图像内容的有效表示。这种方法能够捕捉图像中具有辨识度的局部信息,
在许多图像分类任务中取得了不错的性能。
在本实验中,我们使用SIFT特征实现对场景的分类;

\subsubsection{方法特点}
BoW模型具有以下特点:
\begin{itemize}
    \item \textbf{局部特征表示}:通过提取图像的局部特征来描述图像内容
    \item \textbf{尺度不变性}:使用SIFT等特征提取器保证对尺度变化的鲁棒性
    \item \textbf{旋转不变性}:通过特征描述符的设计保证对旋转变换的鲁棒性
    \item \textbf{空间信息丢失}:忽略特征的空间位置关系
    \item \textbf{计算效率}:特征提取和分类过程相对高效
\end{itemize}

\subsubsection{基本原理}
BoW模型通过以下步骤将图像转换为特征向量:

\begin{enumerate}
    \item \textbf{局部特征提取}:
    使用SIFT算法检测图像中的关键点,
    计算每个关键点的128维特征描述符。
    特征描述符对尺度和旋转变换具有鲁棒性。
    
    \item \textbf{构建视觉词汇表}:
    收集所有训练图像的SIFT特征描述符,
    使用K-Means聚类算法构建视觉词汇表。
    每个聚类中心代表一个视觉词汇,
    词汇表大小(vocab\_size)需要预先设定。
    
    \item \textbf{特征向量生成}:
    对每张图像提取SIFT特征,
    将特征映射到最近的视觉词汇,
    统计每个视觉词汇的出现频率,
    生成vocab\_size维的直方图特征向量。
    
    \item \textbf{分类}:
    使用SVM等分类器进行训练和预测,
    学习特征向量与图像类别的关系。
\end{enumerate}

