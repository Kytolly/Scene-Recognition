\subsection{Tiny Images 实验原理}

Tiny Images 是一种简单而有效的图像特征表示方法,
其核心思想是将高分辨率图像缩放到小尺寸,
并将像素值展平作为特征向量。
Tiny Images常常在场景识别任务中作为基准方法或与其他特征结合使用时,
仍能取得一定的效果。
在本实验中,将Tiny Images与KNN分类器结合,
作为探索不同特征和分类器组合的第一步。

\subsubsection{方法特点}
Tiny Images方法具有以下特点:

\begin{itemize}
    \item \textbf{计算效率高}:特征提取过程简单快速
    \item \textbf{低维表示}:特征向量维度较低(如256维)
    \item \textbf{全局特征}:捕捉图像的整体低频信息
    \item \textbf{对细节不敏感}:对噪声和细节变化具有鲁棒性
    \item \textbf{对几何变换敏感}:对图像的平移、旋转和缩放敏感
\end{itemize}

\subsubsection{基本原理}
Tiny Images 方法通过以下步骤将图像转换为特征向量:

\begin{enumerate}
    \item \textbf{图像缩放}:将原始图像强制缩放到固定的小尺寸(如16x16像素),保留整体结构信息。
    
    \item \textbf{灰度转换}:将彩色图像转换为灰度图像,简化特征表示。
    
    \item \textbf{特征向量生成}:将缩放后的图像像素值按序排列,形成一维特征向量。对于16x16的灰度图像,得到256维向量。
    
    \item \textbf{特征归一化}:对特征向量进行归一化处理,减少光照等因素的影响。
\end{enumerate}